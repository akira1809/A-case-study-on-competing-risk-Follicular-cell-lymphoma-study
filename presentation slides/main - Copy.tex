\documentclass{beamer}[10]
\usepackage{pgf}
\usepackage[danish]{babel}
%\usepackage[english]{babel}
\usepackage[utf8]{inputenc}
\usepackage{beamerthemesplit}
\usepackage{graphics,epsfig, subfigure}
\usepackage{url}
\usepackage{srcltx}
\usepackage{hyperref}
\usepackage[style=ieee]{biblatex}
\usepackage{graphicx}
\usepackage{bm}
\addbibresource{biblatex-examples.bib}

%package for the copyright symbol
\usepackage{textcomp}

%you can repeat this command to define different colors here.
\definecolor{kublue}{RGB}{86,160,209}

\setbeamercovered{transparent}
\mode<presentation>
\usetheme[numbers,totalnumber,compress,sidebarshades]{PaloAlto}
%\setbeamertemplate{footline}[frame number]
\setbeamertemplate{footline}[text line]{%
  \parbox{\linewidth}{\vspace*{-8pt}
  						%Copyright \textcopyright 2015 by Milind A. Phadnis. All rights reserved. \hfill
  						%\insertshortauthor\hfill
  						%\insertpagenumber}
}}

  \usecolortheme[named=kublue]{structure}
  \useinnertheme{circles}
  \usefonttheme[onlymath]{serif}
  \setbeamercovered{transparent}
  \setbeamertemplate{blocks}[rounded][shadow=true]

\logo{\includegraphics[width=1.5cm]{KU-Med-Logo.jpg}}
%\useoutertheme{infolines} 
\title{A case study on competing risks: Follicular cell lymphoma study}
\author{Guanlin Zhang}
\institute{Department of Biostatsitics \\ University of Kansas Medical Center}
\date{07 May 2018}



\begin{document}
\frame{\titlepage \vspace{-0.5cm}
}

\frame
{
\frametitle{Overview}
\tableofcontents%[pausesection]
}

%Introduction corresponding to the introduction section of manuscript
\section{Introduction}

\begin{frame}
	\frametitle{Introduction: Background and Concepts}
	In general, a competing risk situation arises when an individual can experience more than one type of event and the occurrence of one type of event hinders the occurrence of other types of events[\cite{pintilie06}].
	\begin{block}{different ways to express}
		\begin{enumerate}
			\item some described competing risks as the situation in which an individual can experience more than one type of event.
			\item some explained it as the failure to achieve independence between the time to an event and the censoring mechanism.
			\item some defined the concept of competing risks as the situation where one type of event `either precludes the occurrence of another event under investigation or fundamentally alters the probability of occurrence of this other event'.
		\end{enumerate}
	\end{block}
\end{frame}
\begin{frame}
	\frametitle{Introduction: Backgroud and Concepts}
	\begin{block}{examples}
		\begin{enumerate}
			\item bone marrow transplantation for leukemia: death in remission is competing with relapse.
			\item cardiovascular studies: death due to other causes such as cancer are competing with death via cardiovascular disease.
			\item follicular cell lyphoma study: death before relapse is computing with relapse or no response to treatment (our study case).
			\item Incidence of Involuntary pregnancy: Other types of discontinuation of IUD are competing with involuntary pregnancy.
		\end{enumerate}
	\end{block}
\end{frame}
\begin{frame}
	\frametitle{Introduction: Concepts}
	Formal definition: two approaches.
	\begin{block}{latent variable format}
		\begin{enumerate}
			\item $T = \min\{T_1, T_2, \ldots, T_p\}$, $C \in \{0, 1, 2, \ldots, p\}$
			\item multivariate joint survival function: $S(t_1, t_2, \ldots, t_p) = P(T_1 > t_1, T_2 > t_2, \ldots, T_p > t_p)$
			\item subdensity for event type $j$: $f_j(t) = \Big(-\frac{\partial S(t_1, t_2, \ldots, t_p)}{\partial t_j}\Big)_{t_1 = t_2 = \ldots = t_p = t}$
			\item The marginal survivor function for event type $j$: $S_j(t) = S(t_1 = 0, t_2 = 0, \ldots, t_j = t, \ldots, t_p = 0)$
			\item subdistribution function: $F_j(t) = \int_0^t f_j(s)d s$
			\item marginal hazard: $h_j(t) = -\frac{\partial \log(S_j(t))}{\partial t}$
		\end{enumerate}
	\end{block}
\end{frame}
\begin{frame}
	\frametitle{Introduction: Concepts}
	Cause(Type)-Specific hazard:
	$\tilde{h}_j(t) = \lim_{\delta \to 0}\left\{\frac{P(t < T \leq t + \delta, C = j|T> t)}{\delta}\right\}$
	\begin{block}{about cause-specific hazard:}
		\begin{enumerate}
			\item $\tilde{h}_j(t) = \Big(-\frac{\partial \log\Big(S(t_1, t_2, \ldots, t_p)\Big)}{\partial t_j}\Big)_{t_1 = t_2 = \ldots = t_p = t}$
			\item $h(t) = \sum_{j = 1}^p\tilde{h}_j(t)$
		\end{enumerate}
	\end{block}
	Under the assumption of independence for different events, there is $\tilde{h}_j(t) = h_j(t)$
\end{frame}
\begin{frame}
	\frametitle{Introduction: Concepts}
	another approach:
	\begin{block}{bivariate random variable}
		\begin{enumerate}
			\item  record the outcome of competing risks as a bivariate random variable $(T, C)$, $C \in \{0, 1, 2, \ldots, p\}$
			\item Under this approach we could define CIF:  $F_j(t) = P(T \leq t, C = j)$
			\item the subdensity function defined in the previous approach follows $f_j(t) = \frac{\partial F_j(t)}{\partial t} $
			\item $\lim_{t \to \infty}F_j(t) = P(C = j) < 1$ that is why sometimes it is also called sub distribution.
		\end{enumerate}
	\end{block}
\end{frame}
\begin{frame}
	\frametitle{Introduction: Concepts}
	Discussion regaring different approaches:
	\begin{block}{discussion:}
		\begin{enumerate}
			\item For latent failure time approach, we rely on the assumptino of independence between event types. Otherwise marginal distributions could not identify the joint distribution.
			\item However independence is not testable in application when one only observes the first event (examples...).
			\item So people will only consider cause-specific hazard under the bivariate definition.
			\item assumig independence, we can just treat event of interest as our event and all the other event types as censoring. Then the Cox model and parametric AFT model could all come into play.
		\end{enumerate}
	\end{block}
\end{frame}
\section{Follicular cell lymphoma data}
\begin{frame}
	\frametitle{Data Description:}
	\begin{block}{about the data}
		\begin{enumerate}
			\item A hospital database of lymphoma patient data was created at the Princess Margaret Hospital, Toronto, with records dating from $1967$.
			\item A subset of $541$ patients from the data will be used in our case study, all of them identified as having follicular type lymphoma, registered for treatment at the hospital between $1967$ and $1996$, with early stage disease (I or II) and treated with radiation alone (RT) or with radiation and chemotherapy(CMT). 
			\item The goal of this study was to report the long-term outcome in this group of patients.
			\item The response to treatment is given here in a simplified version: CR is complete response and NR is no response.
		\end{enumerate}
	\end{block}
\end{frame}
\begin{frame}
	\frametitle{Data Description:}
	\begin{table}[t]\label{follicular}
	\begin{center}
		\caption{Follicular cell lymphoma data}\label{tab:lymph}
		\scalebox{0.7}{
		\begin{tabular}{ll}
		\hline
			Variable name&Description\\
		\hline
		stnum & Patient ID\\
		\cline{1-2}
		\multicolumn{2}{l}{{\bf Variables assessed at the time of diagnosis}}\\
		age & Age (years)\\
		hgb & Haemoglobin (g/l)\\
		clinstg & Clinical stage: $1 = \text{stage I}$, $2 = \text{stage II}$\\
		ch & Chemotherapy: $\text{Y} = \text{Yes}$, $\text{blank} = \text{No}$\\
		rt & Radiotherapy: $\text{Y} = \text{Yes}$, $\text{blank} = \text{No}$\\
		\cline{1-2}
		\multicolumn{2}{l}{{\bf Outcome variables}}\\
		resp & Response after treatment: $\text{CR} = \text{Complete response}$,\\
		& $\text{NR} = \text{No response}$\\
		relsite & Site of relapse: $\text{L}  = \text{Local}$, $\text{D} = \text{Distant}$, $\text{B} = \text{Local and Distant},$\\
		 & $\text{blank} = \text{No relapse}$\\
		 survtime & Time from diagnosis to death or last follow-up\\
		 stat & Status: $1 = \text{Dead}$, $0 = \text{Alive}$\\
		 dftime & Time from diagnosis to first failure (no response, relapse\\
		 & or death) or last follow-up\\
		 dfcens & Censoring variable: $1 = \text{Failure}$, $0 = \text{Censored}$\\
		 \hline
		\end{tabular}}
	\end{center}
\end{table}
\end{frame}
\begin{frame}
	\frametitle{Data Description:}
	In our study, the event of interest is failure from the disease: either no response to treatment or relapse at any location(coded as $1$), and there is one competing risk that is death without relapse(coded as $2$). The rest of the events falling out of these two categories are considered as censored(coded as $0$).
	\begin{enumerate}
		\item We aim to explore a series of goals, including compare the cause-specific hazard functions for different event types, identify significant covariates for each event type along with proper interpretation, compare the regression coefficients for significant covariates across event types, both globally and individually, identify the most recommended parametric AFT models for each event type by running goodness of fit test, and observing the plot of cumulative incidence function and so on.
	\end{enumerate}
\end{frame}
\section{Methods and Results}
\begin{frame}
	\frametitle{Methods:}
	\begin{enumerate}
		\item[(1)] estimates and tests without covariates 
	\item[(2)] covariate effects via cox models
	\item[(3)] accelerated failure time(AFT) models
	\item[(4)]conditional processes
	\item[(5)] using cumulative incidence functions(CIFs)
	\end{enumerate}
\end{frame}
\subsection{Estimates and tests without covariates}
\begin{frame}
	\frametitle{Estimates and tests without covariates}
	$H_0: h_j(t) = h(t)$ for all j
		\begin{enumerate}
			\item a quick check on frequency: $272$ event, $76$ competing risk, $193$ censoring
			\item formal chi-square test: $\chi^2_1 = 55.2, p < 0.0001$
		\end{enumerate}
    $H_0: h_j(t) = \omega_j h(t), j = 1, 2$
		\begin{enumerate}
			\item log-log survival plot (parallel if proportional)
			\item parametric model by Cox and Oakes(1984):
			 $\log h_j(t) = \alpha_0(t) + \alpha_j + \beta_j t,$ implies a binary logistic model. Coefficient for time is $0$ if proportional. We fit proc logistic and get $\beta = -0.1817, p < 0.0001$, indicating a rejection of the proportionality hypothesis.
			 \item $\beta = -0.1817$ tells us that the hazard for event of interest increases more slowly with time than the hazard for competing risk. Specifically the ratio decrease by about $100\times (1 - \exp(-0.1817)) = 16.7\%$ each year.
		\end{enumerate}
\end{frame}
\begin{frame}
	\frametitle{Estimates and tests without covariates}
	\begin{figure}[t]
	\begin{center}
		\caption{log-log survival plot}\label{log-log}
		\includegraphics[width = 7cm]{log-log-survival.jpg}
	\end{center}
\end{figure}
\end{frame}

\subsection{Covariate effects via cox models}
\begin{frame}
	\frametitle{Covariate effects via cox models}
	We fit the data into cox semi-parametric models (PH model):
	\begin{enumerate}
		\item $h(t) = h_0(t)\cdot \exp(\beta_1\text{age}\_\text{group} + \beta_2\text{hgb} + \beta_3\text{clinstg} + \beta_4\text{chemo}) $
		\item $h_1(t) = h_{1, 0}(t)\cdot \exp(\beta_{1, 1}\text{age}\_\text{group} + \beta_{1, 2}\text{hgb} + \beta_{1, 3}\text{clinstg} + \beta_{1,4}\text{chemo})$
		\item $h_2(t) = h_{2,  0}(t)\cdot \exp(\beta_{2, 1}\text{age}\_\text{group} + \beta_{2, 2}\text{hgb} + \beta_{2, 3}\text{clinstg} + \beta_{2,4}\text{chemo}) $
		\item age is grouped here;
		\item baseline hazards are different.
	\end{enumerate}
\end{frame}
\begin{frame}
	\frametitle{Covariate effects via cox models}
	The goal here:
		\begin{enumerate}
			\item finding significant covariates
			\item test on $H_0: {\bm \beta} = {\bm \beta}_j, j = 1, 2$ with likelihood ratio.
			\item test on $H_0: \beta_{1, i}= \beta_{2, i}, i = 1, 2, 3, 4$ with $T = \frac{(\hat{\beta}_{1, i} - \hat{\beta}_{2, i})^2}{\hat{\text{Var}}(\beta_{1, i}) + \hat{\text{Var}}(\beta_{2, i})}$
		\end{enumerate}
\end{frame}
\begin{frame}
	\frametitle{Covariate effects via cox models}
	covariates effect via cox model: combined events:
	\begin{center}
		\includegraphics[width = 10cm]{cox_all.jpg}
	\end{center}
	also $H_0: {\bm \beta} = {\bf 0}$ has a p value around $0.001$.
%\end{table}
\end{frame}
\begin{frame}
	\frametitle{Covariate effects via cox models}
	covariates effect via cox model: event of interest
	\begin{center}
		\includegraphics[width = 10cm]{cox_event_interest.jpg}
	\end{center}
	also $H_0: {\bm \beta}_1 = {\bf 0}$ has a p value $<0.001$.
\end{frame}
\begin{frame}
	\frametitle{Covariate effects via cox models}
	covariates effect via cox model: competing risk
	\begin{center}
		\includegraphics[width = 10cm]{cox_competing.jpg}
	\end{center}
	also $H_0: {\bm \beta}_2 = {\bf 0}$ has a p value $<0.001$.
\end{frame}
\begin{frame}
	\frametitle{Covariate effects via cox models}
	To test $H_0: {\bm \beta} = {\bm \beta}_j, j = 1, 2$, we have:
	\begin{center}
\begin{tabular}{lll}
	\hline
	model&$-2$log-likelihood&df\\
	combined events & $3888.993$&$4$ \\
	event of interest & $3140.689$  &$4$\\
	competing risk & $719.480$&$4$\\
	\hline
\end{tabular}
\end{center}
The test statistic is
\begin{align*}
	T = 3140.689 + 719.480 - 3888.993 = 28.824 
\end{align*}
with p value:
\begin{align*}
	p = P(\chi^2_4 \geq 28.824) = <0.0001 
\end{align*}
so we reject $H_0$ and conclude that the regression coefficients across different models are different.
\end{frame}
\begin{frame}
	\frametitle{Covariate effects via cox models}
	To test $H_0:  \beta_{1, i}= \beta_{2, i}, i = 1, 2, 3, 4$, we got:
\begin{center}
	\begin{tabular}{llll}
	\hline
	&&test statistic&p value\\
	\hline
	age$\_$group&$\beta_{1, 1}$ vs $\beta_{2, 1}$ &$23.77102$&$<0.0001$\\
	hgb&$\beta_{1, 2}$ vs $\beta_{2, 2}$&$0.0056$&$0.94$\\
	clinstg&$\beta_{1, 3}$ vs $\beta_{2, 3}$&$0.7048$&$0.40$\\
	chemo&$\beta_{1, 4}$ vs $\beta_{2, 4}$&$0.3378$&$0.56$\\
	\hline
	\end{tabular}
\end{center}
We have significant statistical justification to conclude that the coefficient for age group are different between the model for event of interest and the model for competing risk.
\end{frame}
\begin{frame}
	\frametitle{Covariate effects via cox models}
	For the other three covariates, we want to further test whether the three covariates are equal to $0$.
	\begin{center}
	\begin{tabular}{llll}
		\hline
		covariate&null hypothesis&test statistic& p value\\
		hgb&$\beta_{1, 2} + \beta_{2, 2} = 0$&$0.3595$&$0.84$\\
		clinstg&$\beta_{1, 3} + \beta_{2, 3} = 0$&$14.1449$&$<0.01$\\
		chemo&$\beta_{1, 4} + \beta_{2, 4} = 0$&$3.8765$&$0.14$\\
		\hline
	\end{tabular}
\end{center}
So for the rest of the study, we only keep age group and clinical stage in the model.
\end{frame}
\subsection{Accelerated failure time models}
\begin{frame}
	\frametitle{Accelerated failure time models}
	\begin{enumerate}
		\item we fit AFT model to the data, taking age group and clinical stage as covariates.
		\item  for nested models, we use likelihood ratio test
		\item  for other models, we look at AIC and BIC.
		\item for exponential model, SAS gives lagrange multiplier test for  scale parameter: $H_0: \sigma = 1$.
		\item we can also look at output for weibull mode, point estimate and confidence interval for scale parameter.
		\item $\log\frac{P(J = 1|T= t)}{P(J = 2|T = t)} = (\alpha_1 - \alpha_2)\log t + (\beta_{1, 0} - \beta_{2, 0}) + (\beta_{1, 1} - \beta_{2, 1})\text{Age}\_\text{group} + (\beta_{1, 2} - \beta_{2, 2})\text{clinstg}$
	\end{enumerate}
\end{frame}
\begin{frame}
	\frametitle{Accelerated failure time models}
	\begin{center}
\begin{tabular}{lll}
	\hline
	$-2$log-likelihood&event of interest&competing risk\\
	\hline
	exponential&$1825.207$& $436.588$\\
	weibull&$1673.698$& $415.013$\\
	gamma&$1673.695$& $413.606$\\
	log-normal& $1700.212$& $425.910$\\
	log-logistic&$1673.949$& $419.122$\\
	\hline
\end{tabular}
\end{center}
\begin{center}
\begin{tabular}{lll}
	\hline
	AIC&event of interest&competing risk\\
	\hline
	exponential&$1831.207$&$442.588$\\
	weibull&$1681.698$& $423.013$\\
	gamma&$1683.695$& $423.606$\\
	log-normal&$1708.212$& $433.910$\\
	log-logistic&$1681.949$&$427.122$\\
	\hline
\end{tabular}
\end{center}
\end{frame}
\begin{frame}
	\frametitle{Accelerated failure time models}
	\begin{center}
\begin{tabular}{lll}
	\hline
	BIC&event of interest&competing risk\\
	\hline
	exponential&$1844.087$& $455.468$\\
	weibull&$1698.872$&$440.187$\\
	gamma&$1705.162$&$445.073$\\
	log-normal&$1725.386$&$451.083$\\
	log-logistic&$1699.122$&$444.296$\\
	\hline
\end{tabular}
\end{center}
\end{frame}
\begin{frame}
	\frametitle{Accelerated failure time models}
	test on scale parameter of exponential
	\begin{center}
	\scalebox{0.9}{
		\begin{tabular}{cccc}
			\hline
			scale parameter&p value& point estiate & confidence interval\\
			\hline
			event of interest&$<0.0001$&$1.7955$&$(1.6147, 1.9965)$\\
			competing risk&$<0.0001$&$0.6445$&$(0.5451, 0.7621)$\\
			\hline
		\end{tabular}}
	\end{center}
	goodness of fit between weibull and generalized gamma:
	\begin{center}
		\begin{tabular}{lll}
			\hline
				&$\chi^2$ statistic & p value\\
			\hline
			event of interest&$0.003$& $0.96$\\
			competing risk&$1.407$& $0.24$\\
			\hline
		\end{tabular}		
	\end{center}
\end{frame}
\begin{frame}
	\frametitle{Accelerated failure time models}
	$1/(1.796) - 1 = -0.44$: hazard of relapse or no response decrease with time;
	$1/(0.6445) - 1 = 0.55$: hazard of death before relapse increase with time.
	\begin{center}
	\includegraphics[width = 6cm]{smoothed-hazard.jpg}
	\end{center}
\end{frame}
\begin{frame}
	\frametitle{Accelerated failure time models}
	$H_0: (\beta_{1, 1}, \beta_{1, 2}) = (\beta_{2, 1}, \beta_{2, 2})$
	\begin{table}[!htb]
	\begin{center}
		\caption{output for $H_0: (\beta_{1, 1}, \beta_{1, 2}) = (\beta_{2, 1}, \beta_{2, 2})$}
		\includegraphics[width = 4.3cm]{aft_global2.jpg}\includegraphics[width = 4cm]{aft_global.jpg}
	\end{center}
\end{table}
\end{frame}
\subsection{Conditional processes}
\begin{frame}
	\frametitle{Conditional processes}
	$P(T = t, J = j) = P(T =t )P(J = j|T = t)$
	\begin{block}{ }
		Given the event time $T =t$, what is the chance that this is a specific type $J = j$? For the conditional probability, we could use the output of the logistic regreesion to tell us the odds of the event belonging to a specific type. Then for $T = t$ alone, we could just use a cox model for the combined events type.
	\end{block}
\end{frame}
\begin{frame}
	\frametitle{Conditional processes}
	\begin{block}{ }
		The logistic model from equation says that the given that there is a treatment failure(no response, relapse, or death before relapse), patients over age of $65$ only hae about $25\%$ odds of experiencing event of interest(no response, or relapse). 
	\end{block}
	\begin{block}{ }
		The cox model tells that patients who are above age $65$ have about twice as much risk of experiencing failure(by means of either no response or relapse, or death before relapse) . Also  patients who are at clinical stage 2 have about $50\%$ more risk of experiencing treatment failure.
	\end{block}
\end{frame}

\subsection{Cumulative incidence function}
\begin{frame}
	\frametitle{Cumulative incidence function}
	$F_j(t) = P(T \leq t, C = j)$
	\begin{block}{ }
		\begin{enumerate}
			\item  a consistent estimate of the cumulative incidence function is given by $\hat{F}_j(t) = \sum_{k|t_k \leq t} \hat{S}(t_k)\frac{d_k}{n_k}$
			\item $\hat{S}(t) + \sum_j \hat{F}_j(t)\stackrel{p}{\to }S(t) + \sum_j F_j(t) = 1$
		\end{enumerate}
	\end{block}
\end{frame}
\begin{frame}
	\frametitle{Cumulative incidence function}
	\begin{figure}[!htb]
\begin{center}
\caption{cumulative incidence function}
\includegraphics[width=8cm]{cif.jpg}\label{cif}
\end{center}
\end{figure}
\end{frame}
\section{Discussion}
\begin{frame}
	\frametitle{Discussion}
	we have reached the following major conclusion:
	\begin{block}{ }
		\begin{enumerate}
			\item  the hazard rate for relapse or no response is different than the hazard rate for death before relapse
			\item among all the covariates in the data, only age group and clinical stage have significant effect on either of the event type (or combined together)
			\item the most recommended AFT model for either event type is weibull.
		\end{enumerate}
	\end{block}
\end{frame}
\begin{frame}
	\frametitle{Discussion}
	We did not spend much space to discuss the methods assuming dependence between different event types. There are opinions that it may not be all that exciting to assume dependence after all. For any model that incorporates dependence among event types, there is an independence model that does an equally good job of fitting the data. To keep an open mind though, we are aware that there are regression methods based on cumulative incidence functions (Fine and Gray, 1999; Scheike and Zhang, 2008) and might need to explore in the future study.
\end{frame}

\appendix
\section<presentation>*{References}

\begin{frame}[allowframebreaks]
  \frametitle<presentation>{References}
    
\begin{thebibliography}{10}
%    
  \beamertemplatebookbibitems

\bibitem{Allison10}Allison, Paul D. (2010) \emph{Survival Analysis Using SAS: A Practical Guide, 2nd Edition}. SAS Institute Inc., Cary, NC, USA.

\bibitem{Cap94}Caplan, R.J., Pajak, T.F. and Cox, J.D. (1994). Analysis of probability and risk of cause-specific failure. \emph{International Journal of Radiation Oncology.} \emph{Biology, Physics.} {\bf 29}, 1183-1186.

\bibitem{Cox84}Cox, D.R. and Oakes, D.(1984) \emph{Analysis of Survival Data.} London: Chapman$\&$ Hall.

\bibitem{Fine99}Fine, Jason P., and Gray, Robert J., (1999) A proportional Hazards Model for the Subdistribution of a Competing Risk. \emph{Journal of the American Statistical Association} Vol. 94, No. 446,  496-509.

\bibitem{Gel90} Gelman, R., Gelber, R., Henderson, I.C., Coleman, C.N. and Harris, J.R.(1990). Improved methodology for analyzing local and distant recurrence. \emph{Journal of Clinical Oncology}. {\bf 8}, 548-555.

\bibitem{Goo99} T. A., Leisenring, W., Crowley, J. and Storer, B. E. (1999). Estimation of failure probabilities in the presence of competing risks: new representations of old estimators. \emph{Statistics in Medicine, }{\bf 18}, 695-706.

\bibitem{Kalb02} Kalbfleisch, J.D. and Prentice, R.L. \emph{The Statistical Analysis of Failure Time Data.} Hoboken, NJ: John Wiley $\&$ Sons, Inc.

\bibitem{Klein97} Klein, John P. and Moeschberger, Melvin L . \emph{Survival Analysis Techniques for Censored and Truncated Data} Springer. 1997.

\bibitem{Peterson04} P.M., Gospodarowicz, M., Tsang, R., Pintilie, M., Wells, W., Hodgson, D., Sun, A., Crump, M., Patterson, B. and Bailey, D. (2004). Long-term outcome in stage I and II follicular lymphoma following treatment with involved field radiation therapy alone. \emph{Journal of Clinical Oncology}, {\bf 22}, 563S.

\bibitem{pintilie06} Pintilie, Melania. \emph{Competing Risks: A practical Perspective} John Wiley $\&$ Sons, Ltd. 2006.

\bibitem{Scheike08}Scheike, T.H., Zhang, M., and Gerds, T.A., Predicting Cumulative Incidence Probability by Direct Binomial Regression, \emph{Biometrika}, {\bf 95}, 205-220.

%\bibitem{aans}http://www.aans.org/Patients/Neurosurgical-Conditions-and-Treatments/Concussion
%
%\bibitem{agresti} Agresti, A. \newblock{ Categorical Data Analysis.} John Wiley \& Sons, Inc.  3rd Edition,  2013.
%
%\bibitem{barnes} Barnes BC, Cooper L, Kirkendall DT, McDermott TP, Jordan BD, Garrett WE Jr. \newblock{Concussion history in elite male and female soccer players.} Am J Sports Med. 1998;26:433-438.
%
%\bibitem{biausa} http://www.biausa.org/concussion/whatisaconcussion
%
%\bibitem{boden} Boden BP, Kirkendall DT, Garrett WE Jr. \newblock{Concussion incidence in elite college soccer players.} Am J Sports Med. 1998;26:238-241.
%
%\bibitem{collins} Michael W. Collins, Anthony P. Kontons, David O.Okonkwo, et al. \newblock{Concussion is Treatable: Statements of Agreements from the Targeted Evaluation and Active Management(TEAM) Approaches to Treating Concussion Meeting held in Pittsburg, October 15-16, 2015.} Neurosurgery. 2016 Dec; 79(6): 912-929.
%
%\bibitem{eva} Evans R W. \newblock{The postconcussion syndrome: 130 years of controversy.} Semin Neurol. 1994; 14:32-39.
%
%\bibitem{gro} Gronwell DMA, Wringtson, P. \newblock{Cumulative effect of concussion.} Lancet. 1975; 2: 99-997
%
%\bibitem{iss}   National Collegiate Athletic Association..  \newblock{NCAA Injury Surveillance System for Academic Years 1997–2000.}  Indianapolis, IN: National Collegiate Athletic Association; 2000..  
%
%\bibitem{lang} Langois JA, Rutland-Brown W, Wald MM. \newblock{The epidemiology and impact of traumatic brain injury: a brief overview.} J Head Trauma Rehabil. 2006;21:375-378.
%
%\bibitem{main}   Covassin T.,   Swanik C. B.,  and M. L. Sachs.  \newblock{Sex Differences and the Incidence of
%Concussions Among Collegiate Athletes}, Journal of Athletic Training,  38(3): 238-244, 2003.  
%  
%%  \bibitem{Kac56}
%%Mark Kac
%%\newblock{Some stochastic problems in physics and mathematics}. Magnolia Petroleum Co. Colloq. Lect. 2.
%
%%\bibitem{Goldstein51}
%%Goldstein
%%\newblock{On diffusion by discontinuous movements, and on the telegraph equation}. Quart. J. Mech. Appl. math. 4, 129-156.
%
 \end{thebibliography}
\end{frame}
%
\begin{frame}
	\frametitle{Thank You!}
	\begin{center}
		\includegraphics[width = 10cm]{thank-you.jpg}
	\end{center}
\end{frame}
%
%%\section{First section}
%%
%%\frame{
%%\frametitle{Sample Frame Title No. 1}
%%Try some English Lorem ipsum dolor sit amet, consectetur adipiscing elit, sed do eiusmod tempor incididunt ut labore et dolore magna aliqua. Ut enim ad minim veniam, quis nostrud exercitation ullamco laboris nisi ut aliquip ex ea commodo consequat. Duis aute irure dolor in reprehenderit in voluptate velit esse cillum dolore eu fugiat nulla pariatur. Excepteur sint occaecat cupidatat non proident, sunt in culpa qui officia deserunt mollit anim id est laborum.
%%}
%%
%%\subsection{Sample subsection}
%%
%%\frame{
%%\frametitle{Sample Frame Title No. 2}
%%\begin{itemize}
%%\item First item
%%\item Second item
%%\item Third item
%%\end{itemize}
%%}
%%
%%\section{Second section}
%%
%%\frame{
%%\frametitle{Sample Frame Title No. 3}
%%Lorem ipsum dolor sit amet, consectetur adipiscing elit, sed do eiusmod tempor incididunt ut labore et dolore magna aliqua. 
%%\begin{block}{Something important}
%%Einstein's formula
%%$$E=mc^2$$
%%\end{block}
%%}


\end{document}
